\section{Related Works}
Currently, numerous recommender systems have been developed to provide laptop suggestions tailored to user needs. These systems illustrate different approaches to handling user requirements, ranging from content-based and collaborative filtering methods to ontology-based models. As a result, a considerable body of prior research exists in this domain, offering valuable insights and serving as important references for ongoing and future studies.



Bahramian and Abbaspour~\cite{bahramian2015} developed an ontology-based tourism recommender system that applied a spreading activation model to enhance personalization. Their system represented both user preferences and points of interest (POIs) through an ontology, dynamically adapted recommendations with feedback, and addressed cold-start and sparsity problems. The results showed improved diversity and relevance in recommendations.

Ayundhita et al.~\cite{ayundhita2019} investigated a laptop recommender system using an Ontology-Based Conversational Recommender System (CRS). Their approach integrated ontological reasoning with interactive questioning to capture user requirements more effectively. Experimental evaluation showed that the system achieved an accuracy of 84.6\% when tested with functionality requirement questions.


Putra and Baizal~\cite{putra2024} proposed a hybrid laptop recommender system that combined ontology-based filtering with collaborative filtering in a conversational framework. The ontology component mapped functional requirements to technical specifications, while collaborative filtering introduced diversity by leveraging similar user preferences. Their system achieved 93.33\% accuracy, outperforming earlier ontology-only approaches.
