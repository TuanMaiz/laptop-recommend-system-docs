\section{Introduction}
In the era on internet and technology evolution in many aspects of our lives, such as doing works, study, communicating with others. Ever since, laptops have been an essential devices, but choosing a suitable laptop from variety of functionality and specifications is an obstacle to many people, especially with one who is not familiar with technology. Therefore, they need support in determining a laptop that suitable for their daily needs~\parencite{baizal2016}, recommend system is playing crucial role in solving this problem.

Recommender systems play a vital role in suggesting items to users, employing a variety of approaches. Content-Based Filtering (CBF) leverages information from items previously liked by a user, but it often suffers from overspecialization and lacks the ability to capture diverse user interests. Collaborative Filtering (CF), on the other hand, relies on similarities between user profiles, yet it struggles with cold-start and sparsity problems when sufficient user data is unavailable. Ontology-based methods attempt to address these issues by incorporating domain knowledge and semantic relationships between items, but they face challenges such as the complexity of ontology construction and the difficulty of adapting to dynamic user preferences. Hybrid methods combine these techniques, among others, to deliver more accurate and reliable recommendations.

To address the limitations of ontology-based methods, a hybrid approach can be employed to enhance the accuracy of item recommendations. In this study, we propose the development of a Conversational Recommender System (CRS) that integrates ontology-based techniques with collaborative filtering (CF), aiming to achieve higher accuracy and deliver an improved user experience.